\section{Introdução}
%TODO  Gabriel: contextualização do problema, sota, Jirau

O desgaste dos componentes do circuito hidráulico de uma unidade geradora (UG)
em uma usina hidrelétrica acarreta em perda de eficiência na produção de energia e
também paradas de máquinas por defeitos e quebras. Em cenários como o Rio
Madeira, esse desgaste é acentuado devido a alta presença de particulado em
suspensão, sendo necessário a realização de manutenção constante. Entretanto,
como abordado nos relatórios anteriores, a manutenção em si demanda a total
parada da UG, seu total escoamento e, no caso do reparo no revestimento das
pás, até a desmontagem do aro câmara e rotor. O projeto EMMA busca uma solução
para o processo de metalização \textit{in situ}, isto é, revestimento das pás no
ambiente da turbina, diminuindo o tempo de manutenção e, consequentemente, de
máquina parada.

A fim de alinhar os interesses entre a Usina de Hidrelétrica de Jirau e Santo
Antônio e tornar a solução mais geral, como analisado em EMMA-DETAIL, a
escotilha inferior foi indicada como o acesso principal aos componentes da
solução, devido a sua presença em
ambas as usinas, maior facilidade logística e por não necessitar do
desenvolvimento de uma solução totalmente customizada para a aplicação. 

A limitação de acesso, a configuração de espaço confinado nas regiões adjacentes
ao rotor e a logística de transporte apresentaram grande desafio técnico e
guiaram as tomadas de decisão no desenrolar do processo de desenvolvimento da
solução. O conceito final para a metalização das pás das turbinas \textit{in
situ} consiste, então, em um manipulador robótico sobre trilhos modulares e
acopladores magnéticos para fixação. A característica modular do sistema é
fundamental para que seja possível o seu transporte até o interior da unidade
geradora. A facilidade de montagem, assim como sua robustez, foram foco dos
estudos desse documento. 

