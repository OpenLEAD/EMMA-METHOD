\section{Metodologia}
%TODO Gabriel: metodologia da solução, etapas de desenvolvimento (pesquisa de
% mercado, sota, simulações, montagem e etc)


A metodoligia empregada durante o desenvolvimento do projeto EMMA consistiu em
diversas etapas, que alimentavam a seguinte e, caso necessário,
realimentavam uma etapa anterior para refinamento da solução ou alinhamento de
resultados. Primeiramente, no EMMA-SOTA foi realizado uma pesquisa e delimitação
do escopo do problema, etapa fundamental para o completo entendimento do problema e
responsável por direcionar o esboço das primeiras soluções. 

Neste documento será verificada a viabilidade da utilização do manipulador MH12,
confirmando que é possível a total cobertura da pá durante o processo de
revestimento. A análise cinemática, dinâmica serão realizadas com o auxílio de
ferramentas de simulação como a plataforma OpenRAVE. Em seguida a estratégia de
controle e planejamento de trajetória assegura que a movimentação do manipulador
se desevolva de forma contínua em todo o espaço de juntas e que as velocidades e
acelerações máximas sejam respeitadas.

A análise detalhada de cobertura da pá, fornece os requisitos mínimos que a base
mecânica deve obedecer, como forças exercedidas e graus de liberdade necessários
para alcancar todas as posições da base do manipulador. A partir dos conceitos
analisados no EMMA-DETAIL, pode-se comparar diferentes soluções e as vantagens
de desvantagens de cada uma. Foi escolhido o conceito
Prismático-Rotacional-Prismático-Prismático (P-R-P-P) porque mostrou-se a
solução mais viável construtivamente. Foi realizado então o projeto básico da
solução e o dimensionamento dos componentes. Os resultados do dimensionamento
permitirão o detalhamento final da base mecânica, compra de materiais, montagem
e testes.





Solução mecanica

Conceito PRPP trilho primario transporte e trilho secundario posicionamento de
coating

Construcao - material, apoio, ancoragem, montagem e desmontagem

Dimensionamento - elementos finitos, tensoes forcas e momentos

Calibracao - reconhecimento do robo com marcadores
reconhecimento da pa correspondence grouping
simulacao de dados 

