\section{Solução mecânica}
%TODO Estevão: solução final, detalhada  da mecanica com possiveis simulações

A solução de revestimento de uma pá de turbina \textit{in situ} requer um robô
de pequeno a médio porte, capaz de passar pelo limitado acesso da turbina. 
No entanto, a pá da turbina é uma peça com uma grande área a ser coberta pelo
revestimento sendo o robô incapaz de alcançar, de uma só posição, toda a sua
extensão. Assim, é necessário prover ao robô liberdade de posicionamento
para realizar o revestimento em uma pequena região da pá, por posição de
base.

Devido ao peso do manipulador e por questões de segurança, a movimentação deste
no interior da turbina não pode ser uma tarefa manual. Logo, uma base mecânica
deve ser capaz de levar o robô desde a escotilha até a posição ideal
para o revestimento, de forma segura e precisa. O dimensionamento desta base deve levar
em consideração todos os esforços de operação, como: o peso do sistema, as
cargas dinâmicas de movimentação do robô e o empuxo da pistola. 

Foram estudados diversos conceitos para os graus de liberdade providos pela
base mecânica. O estudo destes conceitos estão detalhados no EMMA-DETAIL.

\subsection{Conceito}

A escolha do conceito da solução foi baseado nos graus de liberdade da base
mecânica para permitir ao robô movimentação e alcance necessários para realizar 
o revestimento. 

O conceito aqui aprofundado é denominado P-R-P-P, ou
Prismático-Rotacional-Prismático-Prismático.
A seguir estão descritas cada junta que compõe a base.

\begin{itemize} 

	\item\textbf{Prismática 1:} A primeira junta prismática é formada por um
	trilho, sobre uma estrutura modular. Este trilho é denominado ''trilho
	primário'' e está paralelo ao eixo da turbina, indo desde a escotilha de
	entrada até a região posterior da pá, próxima ao distribuidor.

	\item\textbf{Rotacional:} A segunda junta, rotacional, é formada por um
	eixo apoiado sobre mancais de rolamento, que liga os trilhos ''primário'' e
	''secundário''.

	\item\textbf{Prismática 2:} A terceira junta,
	prismática, é formada por um trilho, denominado ''trilho secundário'', também
	sobre uma estrutura modular. Este trilho é posicionado paralelo ao plano
	transversal da pá, e permitirá ao robô os deslocamentos laterais ao longo de
	toda a extensão da largura da pá. 

	\item\textbf{Prismática 3:} A última junta, prismática, é formada por um
	macaco mecânico tipo sanfona, com curso mínimo de $200~mm$, permitindo ao robô
	maior alcance em altura.
	
\end{itemize}

A figura ~\ref{fig::conceito} apresenta o conceito descrito.

\begin{figure}[H]
	\centering
	\includegraphics[width=0.9\columnwidth]{figs/conceito/conceito_P-R-P-P_01_tags}
	\caption{Conceito P-R-P-P da base mecânica}
    \label{fig::conceito}
\end{figure}

Neste conceito, o manipulador é movimentado pelo trilho primário até a região
próxima a pá. Em seguida é montado o trilho secundário a partir da base onde o
manipulador está fixado. A orientação do trilho secundário é definida pela
movimentação da junta rotacional entre os trilhos primário e secundário. A
partir daí, o manipulador pode ser movimentado ao longo do trilho secundário e
posicionado para o revestimento. Para as regiões de difícil alcance será
utilizada a junta de elevação.

\subsection{Constução}

\subsubsection{Trilho e carrinho}

Para movimentação e posicionamento precisos do manipulador foi selecionado o
sistema de trilho e carrinho de rolamento de esferas recirculantes. 
O trilho selecionado tem o perfil segundo a norma ISO 12090-1 e o carrinho segue
a norma DIN 645-1. 
Estes componentes são próprios para aplicações onde se requer grande capacidade
de carga e precisão de posicionamento.

\begin{figure}[H]
	\centering
	\includegraphics[width=0.7\columnwidth]{figs/construcao/trilho_LLT}
	\caption{Trilho para movimento linear}
    \label{fig::trilho}
\end{figure}

\begin{figure}[H]
	\centering
	\includegraphics[width=0.7\columnwidth]{figs/construcao/carrinho}
	\caption{Carrinho de esferas recirculantes}
    \label{fig::carrinho}
\end{figure}

Estes componentes permitem algumas opções de montagens que variam de acordo com
a aplicação. 
Estas opções vão desde a utilização de um único trilho e único carrinho,
montagens com 1 trilho e 2 carrinhos, até montagens com 2 trilhos e 4 carrinhos,
mostrado na figura~\ref{fig::sist_2por4}.

\begin{figure}[H]
	\centering
	\includegraphics[width=0.9\columnwidth]{figs/construcao/sist_2por4}
	\caption{Montagem com 2 trilhos e 4 carrinhos}
    \label{fig::sist_2por4}
\end{figure}

As cargas promovidas pelo manipulador podem ser elevadas, sobretudo na base, que
reage às cargas distantes do ponto de fixação, causando momentos elevados. 
A vantagem da utilização de mais de um carrinho por trilho é a possibilidade de
anular-se os momentos de reação nos carrinhos, na direção ortogonal ao eixo do
trilho. 
No nosso caso, as cargas são variáveis em sua magnitude e direção.
Por esta razão, a configuração que utiliza 2 trilhos e 4 carrinhos é a mais
indicada, já que os carrinhos ficam livres de reagirem a momentos e as cargas ficam
dividas em mais componentes.

%TODO Estevão: incluir figura 4 carrinhos e 2 trilhos

%---------------------------------------------------------------------
\subsubsection{Perfil de alumínio estrutural} \label{sec::perfil}

A estrutura que servirá de base para o trilho deve ter como prinicipal
característica a modularidade. Devido à geometria variável do ambiente no
interior da turbina, é necessário também que esta estrutura permita 
flexibilidade de montagem dos apoios e ancoragens ao longo do trilho.
Cada módulo que da estrutura será composto pelo perfil estrutural onde
serão fixados o trilho e os acessórios de apoio e ancoragem da estrutura no
ambiente.
Tais módulos devem permitir transporte e montagem manuais, de forma fácil e
rápida.
Por isso, optou-se pelo perfil de alumínio estrutural. Este perfil possui
ranhuras para fixação de componentes padronizados que permite a construção
de uma grande variedade de estruturas funcionais de geometrias simples ou
complexas.

\begin{figure}[H]
	\centering
	\includegraphics[width=0.7\columnwidth]{figs/construcao/aluminio_estrutural}
	\caption{Perfis de alumínio estrutural}
    \label{fig::aluminio_estrutural}
\end{figure}

A figura~\ref{fig::modulo_primario} apresenta um exemplo de um dos módulos do
trilho primário, a partir da montagem deste no perfil de alumínio
estrutural. Este módulo pode ser repetido ao longo do eixo longitudinal do
trilho, formando assim a estrutura completa do trilho. 
Para se ajustar ao ambiente da turbina, há variação apenas do comprimento dos
pés de apoio e dos braços de ancoragem, mantendo-se as mesmas dimensões de todos
os outros componentes.

\begin{figure}[H]
	\centering
	\includegraphics[width=0.9\columnwidth]{figs/construcao/modulo_primario}
	\caption{Perfis de alumínio estrutural}
    \label{fig::modulo_primario}
\end{figure}

%---------------------------------------------------------------------
\subsubsection{Pés de apoio}

Os pés de apoio da estrutura primária têm o objetivo de nivelar o trilho no
ambiente, permitindo que a estrutura possa formar um plano horizontal e paralelo
ao eixo da turbina. Devido às inclinações da superfície do túnel e do
aro-câmara, os pés de apoio devem permitir graus de liberdade que compensem
algum desvio. Além disto, o comprimento de cada ''perna'' da base varia ao
longo da estrutura, sendo necessário permitir uma margem de erro de
montagem a partir de uma regulagem de seu comprimento.

Foi verificado que o maior ângulo formado entre o eixo horizontal
e a superfície da turbina é de aproximadamente $9º$. Optou-se portanto por pés com
interface do tipo rótula, figura~\ref{fig::spindle}, que permitem até $10º$ de
inclinação entre a haste e à base, além de $75~mm$ de regulagem do
comprimento, pela rosca da haste.

\begin{figure}[H]
	\centering
	\includegraphics[width=0.4\columnwidth]{figs/construcao/spindle}
	\caption{Pé com interface rotular entre a haste e a base}
    \label{fig::spindle}
\end{figure}

%---------------------------------------------------------------------
\subsubsection{Ancoragem}

A ancoragem da estrutura é importante para prevenir movimento da base quando o
robô estiver em movimento. Sua principal função é tornar a base rígida o
suficiente para que as deformações elásticas e vibrações da estrutura não
interfiram na precisão do processo de revestimento.

Os braços de ancoragem são constituídos de duas juntas rotacionais em cada
extremidade. Estas juntas permitem que o braço se ajuste a superfície da
turbina, posicionando as bases magnéticas na orientação ideal para o acoplamento
magnético mais eficiente possível. A figura~\ref{fig::ancoragem} apresenta um
dos braços de ancoragem.

\begin{figure}[H]
	\centering
	\includegraphics[width=0.9\columnwidth]{figs/construcao/ancoragem}
	\caption{Braços de ancoragem do trilho primário}
    \label{fig::ancoragem}
\end{figure}

%---------------------------------------------------------------------
\subsubsection{Bases magnéticas}

As bases magnéticas são os elementos de fixação não-permanentes que serão
utilizados para ancoragem da estrutura no ambiente da turbina. Foram realizados
testes para verificação da carga máxima suportada e o resultado foi
satisfatório e apresentado no anexo do EMMA-DETAIL.

As bases magnéticas são equipamentos comerciais cuja principal aplicação na
indústria é o içamento e movimentação de peças metálicas. Este equipamento é
composto por imãs permanentes que são alinhados através de uma alavanca em sua
carcaça. Desta forma, é possível ''ligar'' e ''desligar'' o imã a qualquer
momento.

\begin{figure}[H]
	\centering
	\includegraphics[width=0.5\columnwidth]{figs/construcao/base_magnetica}
	\caption{Base magnética para ancoragem}
    \label{fig::base_magnetica}
\end{figure}

%---------------------------------------------------------------------
\subsubsection{Junta de rotação}

%---------------------------------------------------------------------
\subsubsection{Junta de elevação}

%---------------------------------------------------------------------
\subsubsection{Montagem}

A montagem da base mecânica no interior da turbina é feita a partir de módulos
(submontagens que formam elementos básicos de uma montagem maior). O exemplo de
um módulo da estrutura primária foi apresentado na
figura~\ref{fig::modulo_primario}, na seção~\ref{sec::perfil}. 

Neste conceito, primeiro monta-se a estrutura e trilho
primários, e então, monta-se a base do robô no trilho primário. Esta base contém
os elementos que permitem os graus de liberdade de rotação e prismático
(elevação) e também serve de origem para a montagem da estrutura e trilho
secundários. 

\begin{figure}[H]
	\centering
	\includegraphics[width=0.9\columnwidth]{figs/construcao/EMMA_Base_Secundaria_04}
	\caption{Montagem da base mecânica}
    \label{fig::EMMA_Base_Secundaria_04}
\end{figure}

A figura~\ref{fig::EMMA_Base_Secundaria_04} apresenta uma vista superior da
montagem da base mecânica e a figura~\ref{fig::EMMA_Base_Secundaria_01}
apresenta a montagem no interior da turbina, em uma configuração de operação na
face posterior da pá. Nota-se que é necessária a desmontagem de parte do trilho
primário para posicionar a pá no melhor ângulo em relação à base. Esta é mais
uma vantagem de se ter um conceito modular para a estrutura.

\begin{figure}[H]
	\centering
	\includegraphics[width=0.9\columnwidth]{figs/construcao/EMMA_Base_Secundaria_01}
	\caption{Montagem da base mecânica no interior da turbina}
    \label{fig::EMMA_Base_Secundaria_01}
\end{figure}

%---------------------------------------------------------------------
\subsection{Dimensionamento}

São apresentados os cálculos de dimensionamento dos equipamentos estruturais
da base mecânica.

\subsubsection{Trilho}

Para o trilho, foi utilizada a ferramenta de cálculo
\textit{SKF Linear Guide Calculator} da fabricante SKF (disponível em:
\url{http://www.skf.com/us/knowledge-centre/engineering-tools/skflinearguidescalculator.html}),
que permite o dimensionamento considerando os piores cenários de carregamento,
baseado nas forças e momentos máximos aplicados ao trilho.
O programa analisa toda a gama de dimensões dos trilhos do tipo LLT, fornecendo
o Fator de Segurança obtido para cada opção de tamanho.

O relatório com o resultado de saída do programa encontra-se no anexo A.
%TODO Estevão: Adicionar anexo A: Relatório SKF
O critério escolhido para seleção do trilho baseia-se no Fator de Segurança, que
deve ser maior ou igual a 2 ($FS\geq2$). 

Escolheu-se portanto o trilho LLTHS 45 devido ao resultado dos cálculos
atender ao critério, fornecendo um Fator de Segurança de $2,1$.

%---------------------------------------------------------------------
\subsubsection{Estrutura}

Com o objetivo de verificar as tensões devido às cargas do manipulador e também
verificar os deslocamentos por deformação elástica na base, foi feito um modelo
para análise da estrutura. Este modelo foi resolvido pelo Método de
Elementos Finitos utilizando o programa Autodesk\textregistered{}
Nastran\textregistered{} In-CAD.

A partir dos resultados, pode-se determinar alterações de geometria ou seleção
de materiais, que garantam maior segurança e rigidez da estrutura.
A seguir, são detalhados os parâmetros da simulação de Elementos Finitos.

\textit{Malha:} A análise foi modelada com elementos unidimensionais
de tamanho global de $25~mm$, e ordem quadrática.  E foram atribuidas propriedades
de viga, relativas ao perfil de alumínio estrutural, tais como:
momentos de inércia, momento de inércia polar e área de seção tranversal.

\textit{Material:} O material utilizado é o alumínio EN AW-6060 (norma
DIN EN 573) e foi definido no modelo a partir de suas propriedades físicas e
mecânicas. %TODO Estevão: colocar referência propriedades do material

\begin{center}
\centering
\begin{tabular}{|l|l|l|}
\hline
\textbf{Propriedade}   & \textbf{Valor} & \textbf{Unidade}    \\ \hline
Massa específica       & $2700$           & kg/m³             \\ \hline
Módulo de Elasticidade & $70,0$           & GPa               \\ \hline
Módulo de Cisalhamento & $26,1$           & GPa               \\ \hline
Coeficiente de Poisson & $0,34$           &                   \\ \hline
Limite de Escoamento   & $200$            & MPa               \\ \hline
\end{tabular}
\captionof{table}{Propriedades do EN AW-6060}
\label{tab::prop_material}
\end{center}


\textit{Condições de Contorno:} Para os braços de ancoragem, foram
definidas restrições de translação em todas as direções e rotação livre, na
extremidade da base magnética. Para os pés de apoio, definiu-se restição
de translação apenas na direção Y. A interface entre os trilhos primário e
secundário foi modelada utilizando elementos unidimensionais rígidos, fixados na
mesma posição dos carrinhos de rolamento.

A figura~\ref{fig::contorno} apresenta o modelo de análise da estrutura para o
robô posicionado na face posterior da pá. Estão representados o Sistema de
Coordenadas do modelo, a malha, as condições de contorno e os conectores
utilizados.

\begin{figure}[H]
	\centering
	\includegraphics[width=0.9\columnwidth]{figs/dimensionamento/contorno}
	\caption{Malha e condições de contorno}
    \label{fig::contorno}
\end{figure}

\textit{Carregamento:} O carregamento utilizado para simulação
refere-se às forças e torques máximos atingidos pelo manipulador MOTOMAN modelo MH12, em
sua base. Este carregamento estão de acordo com o manual de instalação e
manutenção do robô como forças e torques de parada de emergência.  
%TODO Etevão: incluir ref. manual MH12

\begin{center}
\centering
\begin{tabular}{|l|l|l|}
\hline
\textbf{Carga}		   & \textbf{Valor} & \textbf{Unidade}   \\ \hline
Força em X		       & 9025           & N	                 \\ \hline
Força em Y			   & -5885          & N                  \\ \hline
Força em Z			   & 9025           & N                  \\ \hline
Momento em X		   & 4120           & Nm                 \\ \hline
Momento em Y		   & 4120           & Nm                 \\ \hline
Momento em Z		   & 4120           & Nm                 \\ \hline
\end{tabular}
\captionof{table}{Forças e torques na base do robô}
\label{tab::carregamento}
\end{center}

As forças foram aplicadas através de um conjunto de conectores rígidos a partir
do ponto que representa o ponto de origem da base do robô. São $4$ conectores
representando a posição de cada carrinho no trilho secundário. Foram testadas
algumas posições do carrinho sobre o trilho para definir o pior caso.
Verificou-se que este ocorre quando o carrinho está a uma distância de
$817,5~mm$ a partir do eixo de rotação entre os trilhos. 
A figura ~\ref{fig::carregamento} mostra a representação do carregamento na
direção resultante.

\begin{figure}[H]
	\centering
	\includegraphics[width=0.9\columnwidth]{figs/dimensionamento/carregamento}
	\caption{Forças e torques na direção resultante}
    \label{fig::carregamento}
\end{figure}


\textit{Resultados:}

Abaixo são apresentados os resultados das simulações realizadas para uma
configuração da estrutura no modelo para revestimento da face posterior da pá. A
figura~\ref{fig::von_mises}, apresenta o resultado das tensões de Von Mises,
demonstrando que o maior encontrado é $5,78~MPa$.

\begin{figure}[H]
	\centering
	\includegraphics[width=0.9\columnwidth]{figs/dimensionamento/von_mises}
	\caption{Resultado de Tensões de Von Mises na estrutura, escala exagerada de
	deformação}
    \label{fig::von_mises}
\end{figure}

O Fator de Segurança pode ser calculado a partir da equação: 

\begin{equation*}
	FS=\frac{\sigma _y}{\sigma _{max}}
\end{equation*}

onde $\sigma_y$ é o Limite de Escoamento do material e $\sigma_{max}$ é a tensão
máxima encontrada. Assim, o Fator de Segurança é $34,6$.

O deslocamento máximo na estrutura está demonstrado na
figura~\ref{fig::deslocamento} e verifica-se que o valor do deslocamento
resultante na base do manipulador foi de $0,47~mm$.

\begin{figure}[H]
	\centering
	\includegraphics[width=0.9\columnwidth]{figs/dimensionamento/deslocamento}
	\caption{Resultado de Deslocamento Resultante na estrutura, escala real de
	deformação}
    \label{fig::deslocamento}
\end{figure}


As forças de reação nos braços de ancoragem são importantes para o
dimensionamento da base magnética. A tabela~\ref{tab::reacao_ancoragem}
apresenta os resultados encontrados em cada braço. A
figura~\ref{fig::mapa_forcas} apresenta a referência de cada braço de
ancoragem, de acordo com a tabela~\ref{tab::reacao_ancoragem}.

\begin{figure}[h!]
	\centering
	\includegraphics[width=0.8\columnwidth]{figs/dimensionamento/mapa_forcas}
	\caption{Referência dos braços de ancoragem para os resultados da
	tabela~\ref{tab::reacao_ancoragem}}
    \label{fig::mapa_forcas}
\end{figure}

\begin{center}
\centering
\begin{tabular}{|l|l|l|l|}
\hline
\textbf{Braço}  & \textbf{Fx {[}N{]}} & \textbf{Fy {[}N{]}} & \textbf{Fz {[}N{]}} \\ \hline
\textbf{Anc\_1} & -1678               & -1966               & 39                  \\ \hline
\textbf{Anc\_2} & -3433               & -4076               & 40                  \\ \hline
\textbf{Anc\_3} & 131                 & 206                 & -569                \\ \hline
\textbf{Anc\_4} & -9556               & 6621                & 6037                \\ \hline
\textbf{Anc\_5} & -1566               & -1436               & 1461                \\ \hline
\textbf{Anc\_6} & 651                 & 160                 & -254                \\ \hline
\end{tabular}
\captionof{table}{Forças de reação em cada ponto de ancoragem}
\label{tab::reacao_ancoragem}
\end{center}

Os resultados \textit{Fx}, \textit{Fy} e \textit{Fz} referem-se às forças
resultantes nas direções \textit{x,y,z} do sistema de coordenadas local de cada
braço de ancoragem. Neste sistema de coordenadas, a direção \textit{x} está
alinhada com o eixo longitudinal de cada braço de ancoragem e as direções
\textit{y,z} são as direções ortogonais a \textit{x}. 
Desta forma, os resultados em \textit{x} negativos indicam tração do braço de
ancoragem e os positivos compressão.

%---------------------------------------------------------------------
\subsubsection{Base Magnética}

O dimensionamento das bases magnéticas está diretamente relacionado ao resultado
das forças resultantes em cada braço de ancoragem, já que são estes os elementos
que representam as restrições no modelo de Elementos Finitos.

Assim, deve-se comparar o resultado da tabela~\ref{tab::reacao_ancoragem} com a
capacidade máxima de carga da base magnética comercial. Para aceitar a base
magnética escolhida, deve-se respeitar as seguintes relações:

 \begin{equation*}
	|F_{x}|\leq\frac{2}{3}*F_{max}
\end{equation*}

 \begin{equation*}
	|F_{y}|,|F_{z}|\leq\frac{2}{3}*\mu*F_{max}
\end{equation*}

Onde \textit{Fx, Fy, Fz} referem-se aos valores obtidos na simulação de
Elementos Finitos apresentado na tabela~\ref{tab::reacao_ancoragem};
$F_{max}$ é a capacidade de carga máxima da base magnética; e $\mu$ é o
coeficiente de atrito entre a base magnética e a superfície da turbina que será
considerado igual a $0,12$.

