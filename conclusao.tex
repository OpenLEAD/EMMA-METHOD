\section{Conclusão}
%TODO TODOS: conclusoes finais

% -.-.-.-.-.-.-.-.-.-.-.-.-.-.-.-.-.-.-.-.-.-.-.-.-.-.-.-.-.-.-.-.-.-.-.-.-.-.-.
%Mecânica
As simulações pelo Método de Elementos Finitos verificaram a rigidez da
estrutura dada, a geometria imposta pelo conceito P-R-P-P e para as opções
comerciais disponíveis de material, perfil de alumínio estrutural, trilho e bases
magnéticas.
Os resultados se mostraram satisfatórios para os os componentes selecionados,
ressaltando que foram considerados casos extremos de operação. A flexibilidade
da estrutura causa erros com ordem de grandeza de $1~mm$, o que não interfere na
qualidade do processo.
As forças resultantes nos pontos de ancoragem permitem dimensionamento e seleção
das bases mecânicas para cada região de ancoragem, não limitando o mesmo tamanho
de base para todos os pontos.
Os resultados de integridade do componentes conferiram Fatores de Segurança
aceitáveis e dentro dos valores recomendados para projetos mecânicos em geral.
% -.-.-.-.-.-.-.-.-.-.-.-.-.-.-.-.-.-.-.-.-.-.-.-.-.-.-.-.-.-.-.-.-.-.-.-.-.-.-.