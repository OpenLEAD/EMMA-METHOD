\subsubsection{Modelagem da superfície}\label{modelagem}

%
%Book
%Geometry and Computing
%Volume 3 2008
%Subdivision Surfaces
%Authors: Jörg Peters, Ulrich Reif
%ISBN: 978-3-540-76405-2 (Print) 978-3-540-76406-9 (Online)

% BEZIER
%Farin, Gerald. Curves and Surfaces for CAGD (5th ed.). Academic Press. ISBN
% 1-55860-737-4.




Existem diversas abordagens matemáticas para descrição de superfícies como:
Parametrização Polinomial; Polinômios em três variáveis; Superfícies de
Bézier; Splines e NURBS (\textit{Non-uniform rational B-spline}); Subdivisão de
superfícies; Malhas poligonais.

Todas essas formas de representar uma superfície, com excessão das malhas,
recaem em alguma instância em uma descrição polinomial. Dentre essas o única
descrição que ocorre de maneira implícita é por polinômios em três variáveis,
descrevendo uma variedade algébrica bidimensional, enquanto as demais são
parametrizações da superfície. 

Por simplicidade, fácil manipulação algébria e implementação, a descrição
puramente polinomial (implícita) foi escolhida como abordagem inicial. De
maneira geral a superfície é descrita como o conjunto solução sobre os
reais da equação polinomial ($f(x,y,z)=0$) de grau $N$, dito grau da
superfície, em $x$,$y$ e $z$:
\[\sum\limits_{i+j+k \leq N}^{} C_{i,j,k}x^iy^jz^k = 0\]

Os coeficientes $C_{i,j,k}$, então, são aqueles que descrevem da superfície.
Devido a restrição do grau do polinômio, o número de coeficientes é
$\binom{N+3}{3}$.

Experimentalmente foi indentificado que um polinômio de quarto grau é suficiente
para aproximar toda uma região de interesse da pá, onde será feito o coating
para uma posição do robô, com erro submilimétrico. Nesse caso, o número de
coeficientes que devem ser identificados é $\binom{7}{3}$, ou seja, 35.

% TODO nome do artigo
Com base no artigo !!!!!!! a passagem da descrição da superfície por nuvem de
pontos para analítica foi feita utilizando a informação da direção da normal à
superfície em cada ponto. Para isso o peso dado a importância da superfície
descrita concordar com a normal em cada ponto foi o mesmo que o peso dado para
que a superfície passe por esse ponto.

Explorando o fato que polinômio são lineares em seus coeficientes, um sistema
superdeterminado, a ser resolvido por mínimo quadrados, foi contruído a partir
do cálculo dos termos do polinômio em cada ponto da nuvem (fazendo $f(x,y,z)=0$)
e da avaliação da normal em cada ponto, que deveria concordar com o gradiente do
polinômio (ou seja $\nabla f(x,y,z) = \overrightarrow{n}$, onde $\overrightarrow{n}$ é a normal no ponto $(x,y,z)$ da nuvem).

%Elael Modelo Polinomial multivarial -> extrapolado para a pá inteira
% PREMISSAS!!


% 
% Splines
% Bézier Surface
% Runge's phenomenon
% Multivariate Polynomial fitting