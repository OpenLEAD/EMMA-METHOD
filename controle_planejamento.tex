\subsection{Planejamento de trajetória}

O conhecimento de todas as regiões que podem ser recobertas pelo robô ajudam na
validação do posicionamento do robô. Porém, ainda é necessário descrever o
caminho a ser percorrido efetuador no espaço de trabalho de maneira a cumprir os
requisitos de coating e o respectivo caminho percorrido no espaço de juntas.

A definição desse caminho é facilitada pelo conhecimento analítico da superfície
da pá que se torna, então, a primeira etapa dessa tarefa. Com essa descrição da
pá são definidos, sobre ela, uma série de faixas que segmenta a pá. Essas
faixas são chamadas de paralelos.

Os paralelos são espaçados de 3 em 3 milímetros para o cumprimento das exigência
de coating e são dispostos de forma a não possuirem intersecção entre si. A
união dos paralelos (considrando uma faixa de 3mm entre eles) recobre todos os
pontos da pá, assim caso percorridos, todos, a pá estará completamente
recoberta.  

O trânsido da ferramenta de coating entre o caminho definido por um
paralelo e outro paralelo é feito por meridianos. Estes são segmentos que
conectam extremidades dos paralelos e não são pensados para cumprir qualquer
exigência de coating. A vávula deve estar fechada e o coating interrompido
durante o tempo que o efetuador percorre todo o meridiano. A razão de existência
dele é meramente definir uma maneira de levar o efetuador de um paralelo a
outro.

\subsubsection{Modelagem da superfície}\label{modelagem}

Existem diversas abordagens matemáticas para descrição de superfícies como:
Parametrização Polinomial; Polinômios em três variáveis; Superfícies de
Bézier (\cite{farin2002curves}); Splines e NURBS (\textit{Non-uniform rational
B-spline}); Subdivisão de superfícies (\cite{peters2008subdivision}); Malhas
poligonais.

Todas essas formas de representar uma superfície, com excessão das malhas,
recaem em alguma instância em uma descrição polinomial. Dentre essas a única
descrição que ocorre de maneira implícita é por polinômios em três variáveis,
descrevendo uma variedade algébrica bidimensional, enquanto as demais são
parametrizações da superfície. 

Por simplicidade, fácil manipulação algébrica e implementação, a descrição
puramente polinomial (implícita) foi escolhida como abordagem inicial. De
maneira geral a superfície é descrita como o conjunto solução sobre os
reais da equação polinomial ($f(x,y,z)=0$) de grau $N$, dito grau da
superfície, em $x$,$y$ e $z$:
\[\sum\limits_{i+j+k \leq N}^{} C_{i,j,k}x^iy^jz^k = 0\]

Os coeficientes $C_{i,j,k}$, então, são aqueles que descrevem da superfície.
Devido a restrição do grau do polinômio, o número de coeficientes é
$\binom{N+3}{3}$. Podendo ser vistos como coordenadas da superfície num espaço
projetivo de dimensão igual ao número de coeficiente,
$\mathbb{P}^{\binom{N+3}{3}}$, em outras palavras a superfície é invariante a
escalamento dos coeficientes.

Experimentalmente foi indentificado que um polinômio de quarto grau é suficiente
para aproximar toda uma região de interesse da pá, onde será feito o coating
para uma posição do robô, com erro submilimétrico. Nesse caso, o número de
coeficientes que devem ser identificados é $\binom{7}{3}$, ou seja, 35.

% TODO nome do artigo
Com base no artigo !!!!!!! a representação analítica da núvem de pontos é um
ajuste de curvas (\textit{curve fitting} \cite{arlinghaus1994practical}) pelo
método dos mínimos quadrados, no qual o peso dado aos vetores normais
das amostras é igual ao peso das amostras. Portanto, o algoritmo calcula os
coeficientes de um polinômio que melhor ajustam (por mínimos quadrados) uma
superfície aos pontos amostrados e aos seus respectivos vetores normais.

Explorando o fato que polinômio são lineares em seus coeficientes, um sistema
superdeterminado, a ser resolvido por mínimo quadrados, foi contruído a partir
do cálculo dos termos do polinômio em cada ponto da nuvem (fazendo $f(x,y,z)=0$)
e da avaliação da normal em cada ponto, que deveria concordar com o gradiente do
polinômio (ou seja $\nabla f(x,y,z) = \overrightarrow{n}$, onde
$\overrightarrow{n}$ é a normal no ponto $(x,y,z)$ da nuvem).

%Elael Modelo Polinomial multivarial -> extrapolado para a pá inteira
% PREMISSAS!!


% 
% Splines
% Bézier Surface
% Runge's phenomenon
% Multivariate Polynomial fitting
\subsubsection{Cálculo dos paralelos}
% como subdividir em segmentos as regiões PREMISSAS!!
Na literatura, há diversas formas de dividir a superfície a ser revestida em
subregiões. Em \cite{from2010off}, por exemplo, um manipulador realiza a pintura
de uma superfície (\textit{spray gun}) cobrindo subregiões de um plano, projeção
da superfície (figura~\ref{fig::pal}). Outra possibilidade é, em funções
paramétricas, realizar uma trajetória semelhante a~\ref{fig::pal} no espaço dos
parâmetricos, cuja transformação (jacobiano) mapeará nos 'cortes' curvos da
superfície.

\begin{figure}[!ht]
	\centering	
	\includegraphics[width=.5\columnwidth]{figs/planejamento/pal.png}
	\caption{Subregiões de uma superfície.}
	\label{fig::pal}
\end{figure}


A superfície descrita na seção~\ref{modelagem} é uma equação implícita, na forma
$f(x,y,z)=0$. Neste caso, as trajetórias a serem percorridas pelo manipulador
podem ser obtidas através da interseção (cortes) entre planos uniformemente
espaçados e a superfície, o que gerará curvas ao longo da superfície. Uma ideia
semelhante e propícia devido à geometria do rotor, é gerar as curvas a partir da interseção
entre esferas e a superfície. As figuras~\ref{fig::interfrontal}
e~\ref{fig::interiso} mostram duas visões de duas interseções entre esferas e
a pá, onde as interseções estão representadas em vermelho.. Os mesmos cortes
podem ser observados entre esferas e o modelo algébrico da pá, em figura~\ref{fig::intergeo}.


\begin{figure}[t!]
	\centering
	\includegraphics[width=0.5\columnwidth]{figs/planejamento/intersecao_frontal.PNG}
	\caption{Interseção esfera-pá, vista frontal.}
	\label{fig::interfrontal}
\end{figure}

\begin{figure}[t!]
	\centering
	\includegraphics[width=0.5\columnwidth]{figs/planejamento/intersecao_iso.PNG}
	\caption{Interseção esfera-pá, vista isométrica.}
	\label{fig::interiso}
\end{figure}

\begin{figure}[t!]
	\centering
	\includegraphics[width=0.5\columnwidth]{figs/planejamento/intersecao_geogebra.png}
	\caption{Interseção esfera-modelo pá.}
	\label{fig::intergeo}
\end{figure}
\input{controle_meridianos}